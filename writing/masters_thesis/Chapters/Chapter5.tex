% Chapter 5

\chapter{Conclusion and Future Work} % Main chapter title

\label{Chapter5	} % For referencing the chapter elsewhere, use \ref{Chapter1} 

\lhead{Chapter 5. \emph{Conclusion and Future Work}} % This is for the header on each page - perhaps a shortened title

%----------------------------------------------------------------------------------------

\section{Summary}

In our tool, there are no handmade features as the network will learn how to extract appropriate features of the raw data. It shows better accuracy as it is trained with the updated viral databases with a good statistical model. This helps us to generalize this model with all genomes and to make a generalized model for sequence classification. We are also able to identify the short viral sequences as LSTM learns from the dependences between the input sequence. 

There is no evidence that these training prokaryotic genomes don't have a viral infection or not. Cleaning the training genomes might give us better accuracy but based on sampling and randomizing prokaryotic fragments, our training data may not contain proviruses. 

For the trained deep learning model, using a sliding window over the input DNA sequence might improve our model, the only drawback of this technique is the slow training and inference time of input sequences, Also using an adaptive learning rate decaying over time steps during the training might improve the model performance, but will need more tuning over the input data.

\section{Conclusion}

This attentional neural deep learning network was able to achieve state of the art results on viral identification from high throughput sequences. Our approach is able to classify short fragments as well. Experimental results validate our approach for identification with an accuracy of more than 83\%. According to these results, Our model would help us in understanding viruses in various microbial communities and discovering new species of viruses.

\section{Future Work}


we will work on this problem in the future ISA. 