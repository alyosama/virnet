% Chapter 1

\chapter{Introduction} % Main chapter title

\label{Chapter1} % For referencing the chapter elsewhere, use \ref{Chapter1} 

\lhead{Chapter 1. \emph{Introduction}} % This is for the header on each page - perhaps a shortened title

%----------------------------------------------------------------------------------------

\section{Metagenomic analysis}

\subsection{Definition}

{M}{etagenomics} is an analysis of the genetic information of the collective genomes of the microbes within a given environment based on its sampling regardless of cultivability of the cells. \cite{izard2014metagenomics}. There is a minor population of microbial organisms identified due to the difficulty in studying them using pure culture isolation. This methodology has been constrained to less than 1\% of host cells and is biased to certain species \cite{labonte2015single}.
% https://www.the-scientist.com/daily-news/most-gut-microbes-can-be-cultured-33581
Metagenomic analysis process demonstrates a promising understanding of different microorganisms. It answers some questions about the identity of microorganisms in the collection and their potential functional characterization.

\subsection{Microorganism}


Microorganisms are found everywhere on earth, and they are critical in our survival. This study, our interest is in prokaryotic microorganisms (e.g. bacteria and archaea) and viruses. Bacteria are unicellular and microscopic organisms that reproduce by binary fission. On the other hand, viruses are typically submicroscopic consists of genetic materials either DNA or RNA surrounded by a protective coat of proteins and can only replicate inside living host cells. They lack metabolic enzymes and translational machinery such as ribosomes for making proteins. There are 200 to a few thousand genes in the bacterial genomes, while the tiniest viral genomes have only three genes and the largest have up to 2000 genes.


\section{Viruses}


\subsection{Definition}

Viruses have an impact on different microbial communities, and virus-host interaction can change many ecosystems such as human health and aquatic life. Phages or bacteriophages are viruses that infect bacteria. Furthermore, phages are abundant in different microbiome communities. The viral infection starts when virus binds to a host cell and its genome integrates with the host cell genome. The integrated viral DNA is called a provirus. It is reasonable to think that isolated viruses are just package of genes moving from one host cell to another.

\subsection{Importance in clinical and environment}

very very important.

\subsection{Identification}

 Scientists are using isolation and culturing techniques to study viral diversity and viral-host interactions in microbial communities. Those techniques have many limitations because there is no universal marker gene for viruses. The sequenced viruses in NCBI RefSeq database constitute approximately 5\% of known species of prokaryotic organisms \cite{roux2015viral}.

\section{Next Generation Sequencing}

High throughput sequencing technology is used for metagenomic studies which can generate large number of read  sequences of microorganisms. The expected read length is up to 600 bp and the number of generated reads per run is up to 15 million approximately based on the sequencing platform and the library preparation methods \cite{allali2017comparison}. We can sequence mixture of prokaryotic cells and viruses in complex microbial communities in a cultivation-independent process. Sequencing of microbial samples shows contamination of viral sequences within prokaryotic population. A study found 4-17\% virus sequences in human gut prokaryotic metagenomes \cite{minot2011human}. Moreover, cellular contamination is quite frequent even with a careful purification of viral particles, and this is one of the main reasons why we need a tool that can differentiate between bacterial and viral sequences.

\subsection{Sequencer Tools}

different tools of HTS
\subsection{Data types}
different types of data

\subsection{Sequence identification in HTS}


The broadly adopted technique to know who is in metagenomic data is to assemble the high throughput reads to contigs then search against a known genomic database using sequence alignment method in order to infer the type of microorganisms and the existence of species in a metagenomic sample. This approach is minimal because it only detects viruses almost related to those we already know. It is reported that about 15\% of viruses in the human gut microbiome and 10\% in the ocean have similarity to the known viruses \cite{ren2017virfinder}. 

\section{Machine learning}

Machine learning approaches have been used to classify and cluster data based on extracted features. The deep neural network is one of machine learning methods that are considered as a state of the art category for general classification problems. Deep learning shows significant improvements in several artificial intelligence tasks for example image classification, speech recognition, and natural language processing. Moreover, It shows significant results with genomic data \cite{angermueller2016deep}. %\\

\section{Our Contribution}

In this paper, we introduce a deep sequence model, VirNet, to identify viral reads from a mixture of viral and bacterial sequences and purify viral metagenomic data from bacterial contamination as well. That will guide us to identify new viruses and potentially perform functional characterization. Additionally, it will answer many mysteries related to our understanding of their functionality and diversity in the ecosystem.

