% Chapter 2

\chapter{Related Work} % Main chapter title

\label{Chapter2} % For referencing the chapter elsewhere, use \ref{Chapter1} 

\lhead{Chapter 2. \emph{Related Work}} % This is for the header on each page - perhaps a shortened title

%----------------------------------------------------------------------------------------

There has been extensive prior work on viral identification. Recent work has focused on identifying phages in bacterial genomes. Several methods have used similarity search by sequence alignment with the reference genomes in order to find viral contigs. Most of the recent tools fall under three categories based on the sample structure such as:
\begin{enumerate}
	\item phages from prokaryotic genomes
	\item viral sequences in mixed metagenomic datasets
	\item phages and viral sequences. 
\end{enumerate} 

\section{Similarity tools}

There are many software packages to find phages from prokaryotic genomes such as Phage\_Finder \cite{fouts2006phage_finder}, Prophinder \cite{lima2008prophinder}, PHAST\cite{zhou2011phast}, and PhiSpy \cite{akhter2012phispy}. These tools are using similarity search to known virus databases using features such as genes. Some of them such as PhiSpy integrates other features such as  unique virus k-mers, AT and GC skew, protein length and transcription strand direction. They have many limitations as they failed to detect viral sequences in metagenomic data as the databases are outdated, limited and don't represent viral diversity in the environment. Moreover, It is not optimized to process a large number of contigs \cite{roux2015virsorter} as they depend on alignment and homology processing limitations. 

The second category is able to detect viral sequences in mixed metagenomic datasets such as VIROME \cite{wommack2012virome} and MetaVir\cite{roux2011metavir}. They are using similarity search with the databases same as the first category. Additionally, they are searching against proteins. There are more packages such as DIAMOND \cite{buchfink2014Diamond} or Centrifuge \cite{kim2016centrifuge} which are much faster and efficient than the former tools for microbial classification. Again, The limitation of this approach is using limited known reference databases. 
%----------------------------------------------------------------------------------------

\section{Statistical tools}


The third category of software packages such as VirSorter \cite{roux2015virsorter} is able to detect phages and viral sequences. VirSorter is using similarity search to viral databases and integrates other features related to analysis of sequence genes such as enrichment of viral-like genes, enrichment of uncharacterized genes and viral hallmark gene. These features make the identification more accurate but it still suffer limitations. One of the limitations is the requirements of having at least 3 genes within the contig because the smallest virus genome contains 3 genes because the smallest virus discovered has 3 genes only so it has the same limitations as previous techniques because of using homology strategy. Moreover, it cannot work with short fragments or contigs and it is very slow in processing metagenomic datasets. 

Recently VirFinder \cite{ren2017virfinder} applied machine learning techniques. VirFinder is a statistical method based on the logistic regression model. It uses the K-mer feature which is considered as a discrimination feature for different sequence problems. It shows a great success with short sequences too and They found a great k-mer similarity score with viruses within other prokaryotic genomes. 

% need to write deep learning techniques in fragments classification

In this paper, we are using deep learning techniques which is much more suitable to sequence problems and also shows significant improvements to other current machine learning models. In deep neural networks, the model will extract the most appropriate features during training which lead to better identification accuracy and sensitivity. 

