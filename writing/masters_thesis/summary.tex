% Settings for Preliminary Pages
\newpage
\thispagestyle{empty}
%\pagestyle{plain} % No headers, just page numbers
%\setcounter{page}{2}

%-----------------------------------------

%-----------------------------------------
\begin{center}\huge \textbf{Thesis Summary}\end{center}


\begin{center}
\underline{\textbf{Summary}}
\end{center}
%

Metagenomics shows a promising understanding of function and diversity of the microbial communities due to the difficulty of studying microorganism with pure culture isolation. Moreover, the viral identification is considered one of the essential steps in studying microbial communities. Several studies show different methods to identify viruses in mixed metagenomic data and phages in host genomes, using homology and statistical techniques. These techniques have many limitations due to viral genome diversity. In this work, we propose a sequence deep neural model for viral identification of metagenomic data. For testing purpose, we generated fragments of viruses and bacteria from RefSeq genomes with different lengths to find the best hyperparameters of our model. Then, we simulated both microbiome and virome high throughput data from our test genomes dataset with aim of validating our approach. 
Finally, we applied our tool to a case study of two types of metagenomic data such as Roche 454 and Illumina.
We compared our tool to the state-of-the-art statistical and popular tool for viral identification and found the performance of VirNet much better regarding accuracy and speed on the same testing data. This tool will help us in growing our insights to natural viruses of microbial communities.

%
\begin{flushleft}
\underline{Chapter 1}
\end{flushleft}
%
%
%
%
\begin{flushleft}
\underline{Chapter 2}
\end{flushleft}
%
%
%
\begin{flushleft}
\underline{Chapter 3}
\end{flushleft}
%
%
%
\begin{flushleft}
\underline{Chapter 4}
\end{flushleft}
%
%
%
%
\begin{flushleft}
\underline{Chapter 5}
\end{flushleft}
%
%
%
\begin{flushleft}
\underline{Chapter 6}
\end{flushleft}
%
%
%
\begin{flushleft}
\underline{Chapter 7}
\end{flushleft}



\begin{flushleft}
\large

Key words: bioinformatics, classification, deep learning, metagenomics
\end{flushleft}



